% ************************** Thesis Abstract *****************************
% Use `abstract' as an option in the document class to print only the titlepage and the abstract.

\nomenclature[]{AI}{Artificial Intelligence}

\begin{abstract}
% maybe should modify this to include some notes about how multi-agent systems are becoming more ubiquitous in society and that they naturally solve many research problems.

Systems utilitising autonomous agents are becoming increasingly pervasive in today's society, garnering commercial interest and research funding in a variety of domains ranging from home automation to undersea exploration. This has stemmed from a resurgence in interest in Aritifical Intelligence over the last number of years. Globally, we are starting to move towards an age of automation through physical and software systems that exhibit redundancy, modularity and robustness. Research into how to induce intelligent decentralised behaviour in such systems will be key to their development.\par
Autonomous systems that can be operated remotely are highly suitable to disaster scene management, due to their highly dangerous and uncertain nature. This thesis outlines the design and implementation of a team of autonomous robots that implement a multi-phase disaster scene management plan.
%the goal of which is to solve a real-world problem in the domain of disaster scene management. 
The problem domain involves robotic aerial vehicles that have sensors and actuators to interact with their environment. Our framework is described abstractly and can be used with different physical agents, with few restrictions on the capabilities and specification of the agents.\par

First, the design and development of a purpose-built high-fidelity simulation environment using a game engine is outlined. This simulation environment has been used extensively in the research project, ROCSAFE, that motivated the work in this thesis. The ROCSAFE project is discussed in Section \ref{sec:ROCSAFEBG}. The simulation environment has helped to address the problem of generating data to carrying out training, testing and validation of systems related to the management of scenarios that are perilous in nature. It has proven a valuable tool for prototyping the research work that has been done for this thesis.

We then discuss the problem of developing an autonomous system to aid the management of a disaster scene. The problem can be broken down into two key sub-problems. The first is a surveying problem, whereby a swarm of aerial vehicles need to use sensors to record data at each point in a discretised region defined by a bounding polygon. This is a standard early phase of a forensic examination of a crime scene and the data gathered from this survey can be used to guide strategies used during subsequent phases of the disaster management process. %Examples of how this information can be used are presented, such as using structure-from-motion to create a textured point cloud that can then be used to plan a safe path for forensic evidence recovery by a ground vehicle.
\par

The second is a stochastic search problem, where multiple agents are used to pinpoint the location of a target, or multiple targets, in a bounding region. The term "target" is used to mean anything that can be sensed by the agents, for example a source of radioactive material. It is assumed that agents are fitted with sensors and actuators and can move around the bounding region freely. Sensor readings are assumed to have some inherent noise, and a stochastic approach is presented which takes this fact into account. Results and analysis of the framework is presented to give insight into how it can be used to formulate search control strategies that optimise some realistic objectives.% Constraints present in the real world are enforced, such as limited communication between agents.



%the developed system is tested using a purpose built simulation environment, which is intended to be a high-fidelity representation of a forensic examination scenario and results are presented.\par

%Results show that the system developed ...
%\break
%List of things that need to be changed
%\begin{enumerate}
%    \item Sometimes mistakenly used 'multinomial', change where appropriate
%    \item Change small n to big N when referring to grid
%    \item discuss how varying height can be incorporated to sensor model
%    \item Check references are correct and fix formatting
%    \item Sometimes use "source" when should be "target"
%\end{enumerate}

\end{abstract}
