\workinprogress

A problem explored in this thesis is an instance of target detection using a system of aerial robots. This problem initially gained traction in the literature with the work of Koopman \cite{KoopmanTheoryOfSearchTargetDetection} and has received much attention since. More recent papers have framed the problem in terms of agents and multi-agent systems, caused the resurgence of AI. A common feature in the problem definition in the literature is that the system of agents are assumed to work in a partially observable environment\cite{Symington2010ProbabilisticUAVs}, \cite{Chung2008Multi-agentFramework}, \cite{WongMulti-vehicleTargets}. Many recent approaches follow the groundwork laid by Elfes\cite{ElfesUsingNavigation}, whereby an "occupancy field" is used to represent the distribution of the target over the cells of a spatial lattice. Elfes describes an Occupancy grid as a "probabilistic tesselated representation of spatial information". This is in contrast to previous approaches that typically used geometric models of the world, which enforced strong domain-specific dependencies. The occupancy field framework 
\par