%outlines the background behind the sequential probability ratio test.

This section assumes that the reader has a basic familiarity with the terminology and techniques of hypothesis testing in statistics. 

A statistical test is a mechanism for making quantitative decisions about a process. Statistical tests are usually used in conjunction with a \textit{statistical hypothesis}, which is a testable hypothesis based on observations of a process that is modelled using random variables. There are many texts that outline the statistical tests that can be used to make a decision between accepting or rejecting the null hypothesis, $H_0$ \note{do I need to list them}. The usual context in which this occurs is one in which the data has been collected in advance and the sample size is fixed and known.\note{An example might be of value here} \par

There exists a branch of statistical hypothesis testing called \textit{sequential hypothesis testing}, which is used when the sample size is not fixed in advance. This means the decision process goes beyond deciding whether to accept or reject the null hypothesis, but to either
\begin{itemize}
    \item Accept the hypothesis being tested, $H_0$.
    \item Reject the hypothesis being tested, $H_0$
    \item Continue the experiment by making a further observation.
\end{itemize}

It is clear that samples are gathered as long as 1). or 2). above are not chosen, which intuitively corresponds to the notion of making an informed decision, where it is desirable to ensure that enough data has been gathered to draw a meaningful conclusion. In the sequential testing paradigm, two kinds of error may be committed, as with the non-sequential case: we may reject the null hypothesis when it is true (commit type \rom{1} error) or we may accept the null hypothesis when some alternative hypothesis is true (type \rom{2} error)

This is explained in further detail by \citeauthor{Wald1945SequentialHypotheses} in \cite{Wald1945SequentialHypotheses}. 


When the decision process involves deciding whether to 1). Accept 

is of a slightly different nature to the usual statistical framework in that it is used 

The Sequential Probability Ratio Test (SPRT) is a statistical test that 