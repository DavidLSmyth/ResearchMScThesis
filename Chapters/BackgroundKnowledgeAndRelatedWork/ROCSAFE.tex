\section{The ROCSAFE Project}\label{sec:ROCSAFEBG}

\note{not sure how deal with double acronym in ROCSAFE}
\textbf{R}emotely \textbf{O}perated \textit{\textbf{C}hemical, Biological, Radiation, Nuclear, explosive (CBRNe)} \textbf{S}cene \textbf{A}ssessment and \textbf{F}orensic \textbf{E}xamination (ROCSAFE) is an EU Horizon 2020 project. Horizon 2020 is an EU Research and Innovation Program, with almost €80 billion in total funding allocated from 2014 to 2020. The ROCSAFE project comes under the \textit{Secure societies - Protecting freedom and security of Europe and its citizens} programme, whose objective is \textit{"to foster secure European societies in a context of unprecedented transformations and growing global interdependencies and threats, while strengthening the European culture of freedom and justice".}
\href{https://cordis.europa.eu/programme/rcn/664463/en}{ }\footnote{\href {https://cordis.europa.eu/programme/rcn/664463/en}{https://cordis.europa.eu/programme/rcn/664463/en}}

Specific details of the ROCSAFE project can be found at the \href{https://cordis.europa.eu/project/rcn/203295/factsheet/en}{Horizon 2020 website}\footnote{\href {https://cordis.europa.eu/project/rcn/203295/factsheet/en}{https://cordis.europa.eu/project/rcn/203295/factsheet/en}} 
and 
\href{http://rocsafe.eu/}{ROCSAFE website.}\footnote{\href {http://rocsafe.eu/}{http://rocsafe.eu/}}. We summarise the relevant details here, based on these primary sources. The high-level goal of ROCSAFE is to fundamentally change how CBRNe events are assessed, given that current practices require personnel to enter hazardous areas with unquantified risks. The project does not aim to provide a first response to CBRNe events; rather it intends to provide support to the forensic phase of the investigation, which is chiefly concerned with the collection, preservation, and scientific analysis of evidence during the course of an investigation, ideally leading to the ability to present admissible evidence during a criminal investigation. 
%It is worth noting that forensic investigations are not particularly time-sensitive; for large-scales scenarios they can typically last for months or years.\note{would be good to find a source for this}\par


CBRNe events are exceedingly difficult to prepare for because they are so rare and diverse, with the consequence that limited data from real events is available to use as a reference. Despite this, many of the procedures related to forensic investigations are well-defined, in order to preserve the chain-of-custody of evidence. The work done in this thesis was identified at the inception of the project to aid: 
\begin{enumerate}
    \item The initial phase of the forensic investigation, in which a survey of the scene is carried out in order to provide high-level information related to the examination area to the crime scene manager.
    \item The identification and localization of forensic evidence that can be detected using sensors developed as part of the project.
    \item Prototyping, testing and validation of some of the technologies related to this project. This was done by designing a high-fidelity simulation environment from scratch.
\end{enumerate}



