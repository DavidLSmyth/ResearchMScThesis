%Want to introduce the problem to allow for a discussion that makes sense. Ideally will frame a problem so that it suggests that having a high-fidelity simulation environment would be very useful for the project and in general.
According to \citeauthor{Shannon1998INTRODUCTIONSIMULATION}, a software simulation can be defined as \textit{ the process of designing
a model of a real system and conducting experiments with
this model for the purpose of understanding the behavior of
the system and /or evaluating various strategies for the
operation of the system}.Simulations written with software are usually created in order to gain insight into the system's dynamics and to evaluate results of using a method intended for use in the real world. This is usually because the may be too time-consuming or expensive to test in the real world. Software simulations are increasingly used for a wide variety of tasks, from planning new roads to alleviate traffic \cite{Pell2017TrendsSimulation} to developing self-driving vehicles \cite{Dosovitskiy2017CARLA:Simulator}]. 

Designing simulations allows the developer of an agent-based system to abstract away details of the real-world conditions that the agents will act in and focus on the salient aspects that the agents are concerned with, in order to prototype, train, test analyse and validate. Simulations are typically of high value when data required to design system:
\begin{itemize}
    \item Costs a lot of money to generate.
    \item Is dangerous to generate.
    \item Is time consuming/laborious to generate.
    \item The system has a well-known stochastic element which makes generating a sufficiently large sample size difficult.
\end{itemize}
The domain which motivated the development of the technologies proposed in this thesis satisfies all of the above requirements, which naturally motivates the design of a simulation environment. \par

Simulations have been used extensively to model complex systems since the invention of the modern computer. Simulation technology has gained traction since the proposal of the Markov Chain Monte Carlo method by Stansislaw Ulam and John Von Neumann in the late 1940s \cite{Robert2011AIncomplete}. Key properties of a simulation 

%Intelligent agents are frequently designed to solve problems which usually would be on or more of: labour intensive, monotonous, repetitive or time consuming. In order to accomplish these tasks, agents may need to use data from 

Simulation environments are used to solve a number of different problems when using intelligent agents.

A consequence of this is that any systems that are developed to provide support can be difficult to evaluate and validate. To address this, we developed a high-fidelity simulated environment as part of the research involved in this thesis, which preserves the critical aspects of CBRNe incidents without presenting any risk of exposure to the dangerous elements of such scenes in the real world. The following published works related to the development of the simulation environment are discussed in this chapter: \citet{Smyth2018AInvestigation}, \citet{Smyth2018UsingDrones}.\par


Virtual environments designed using games engines have previously been used to gather data to train models for a range of applications \cite{1608.02192}\cite{uav_benchmark_simulator} and software packages have been written that can generate photo realistic images from games engines\cite{1609.01326}. There have also been attempts made to develop models of physical systems that simulate potentially dangerous environments \cite{4625089}. To the best of our knowledge, there are no prior applications using games engines to model critical incidents for the purpose of developing analytical tools in a virtual setting that will then transfer to real-world deployment.