The contributions of the work done for this thesis are summarised:
\begin{itemize}
    \item We designed a high-fidelity simulation environment, which was open-sourced for public use with an MIT Licence at the linked \href{https://github.com/NUIG-ROCSAFE/CBRNeVirtualEnvironment/releases}{github repository}\footnote{\href {https://github.com/NUIG-ROCSAFE/CBRNeVirtualEnvironment/releases}{https://github.com/NUIG-ROCSAFE/CBRNeVirtualEnvironment/releases}} https://github.com/NUIG-ROCSAFE/CBRNeVirtualEnvironment/releases.
    \item We designed and open-sourced a software framework to solve the problem of target detection in a hazardous environment, with discrete implementations addressing the problem of localising objects and sources of radiation.
    \item We designed and open-sourced software that can generated uniformly spaced grid points over a polygonal region of space.
    \item We designed and open-sourced software that can generated uniformly spaced grid points over a polygonal region of space.
    \item We designed and open-sourced software that can efficiently generate routes for multiple agents to cover uniformly-spaced grid.
    \item We designed and open-sourced a user interface which allows the user to specify the bounding points of a polygonal region using the WGS84 coordinate system. Using the grid generation  software, the user can generate a grid over this region. Using the agent-routing software it is then possible to request for routes be generated for multiple agents to execute in a real or virtual environment. If the virtual environment developed above is chosen, the user can optionally request for data-gathering tasks be carried out while visiting each grid point.
\end{itemize}

Publications related to this research are listed:
\begin{enumerate}
    \item \bibentry{Smyth2018AInvestigation}
    \item \bibentry{Smyth2018UsingDrones}
    \item \bibentry{Smyth2018ASupport}

\end{enumerate}