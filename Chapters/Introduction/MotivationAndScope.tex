%\nomenclature[F]{CBRNe}{Chemical, Biological, Radiological, Nuclear, explosive}

Technologies based on systems of intelligent agents have received an increasing amount of interest in both research and industry in recent years. Mobile robotics applications such as Simultaneous Localisation and Mapping (SLAM) \cite{Thrun:2005:ProbabilisticRobotics} and target detection and tracking have received substantial interest in particular, with a focus on the multi-robot setting \cite{Saeedi2016Multiple-RobotReview}, \cite{Robin2016Multi-robotSurvey}. 
%This is due to factors such as increased computing power, decreasing lightweight sensor costs, more accurate models of robot environments and improved simulation technologies.
Application domains have naturally gravitated towards problems that would pose hazards to physical human presence \cite{Muller2014ApplicationSurvey}. The work outlined in this thesis has been funded and motivated by a EU Horizon 2020 research project called ROCSAFE (Remotely Operated Chemical, Biological, Radiological, Nuclear(CBRN) Scene Assessment and Forensic Examination) \cite{Bagherzadeh2017ROCSAFE:Incidents}. The goal of the project is to fundamentally change how CBRNe scenes are assessed by ensuring the safety of crime scene investigators. The project uses autonomous robots to reduce the need for investigators to enter high-risk areas when their job requires the determination of threats and gathering of  forensic evidence.\par

The scope of this thesis lies in developing intelligent autonomous agents in the realm of mobile robotics, in order to carry out tasks that are of value in scenarios that would present dangers to humans. A system of heterogeneous aerial vehicles is considered, with sensing capabilities as well as high-level navigation capabilities. The focus is primarily on designing software that implements strategies to efficiently carry out area surveillance and target localisation, independent of the hardware on board the aerial vehicles. Realistic constraints such as operational speeds and battery capacity are considered. The use of high-fidelity 3-D simulation environments has also been investigated, due to the highly complex nature of hazardous real-world scenarios that require surveillance. The work done in this thesis has complemented much of the research that has been carried out in parallel in the ROCSAFE project, for example image processing modules.\par