We identified two main research questions to be investigated for this thesis. They are as follows:
\begin{enumerate}

    \item Can an agent-based software system be developed to run on a system of heterogeneous autonomous aerial vehicles to aid the tasks of scene surveying and target localisation in a hazardous environment? 
    
    \item Can a high-fidelity simulation environment be designed to simulate a hazardous environment, which can generate realistic data to aid the process of prototyping and developing AI systems to be deployed in the real world?
    
\end{enumerate}

Both research questions were derived from the aims of the ROCSAFE project. \href{https://www.nuigalway.ie/rocsafe/research/}{Deliverable 2.3}\footnote{\href {https://www.nuigalway.ie/rocsafe/research/}{https://www.nuigalway.ie/rocsafe/research/}} of the project provides a gap analysis in the management of the forensic phase of a crime scene. 
%the document is based on....
Among the topics that were addressed were
\begin{itemize}
    \item Scene Assessment: The document identifies a gap in scene assessment capabilities at the time of publication: 
"\textit{Currently only limited means are available for remote scene assessment. These are slow and uncoordinated.}". The desired solution that the ROCSAFE project was intended to provide is stated as "\textit{To provide a drone-based overview and reconnaissance of the scene and the impacted area and to provide a means to feed that data into a CDM system}". The document states "\textit{Scene assessment and crime scene planning is significantly aided by photography and mapping.  A critical step in an evidence collection plan is rapid and safe overview of the scene.}" as motivation for this solution.

    \item Detection, Identification and Monitoring (DIM): Deliverable 2.3 outlines the requirements of DIM in relation to forensic analyses:
    "\textit{The DIM concept requirements are based on out-ruling possibilities one by one. This should be relied on by ROCSAFE and followed by ROCSAFE. The DIM process is not concerned with the detection equipment – it is the process of identification that is highlighted}". It is noted that for the ROCSAFE project, carrying this out using remote vehicles is a key output: "\textit{In order to reduce risk and to determine areas of contamination. Remote means are required, and which should follow the DIM sequence of action.}". The document states that a proposed solution should incorporate the following: "\textit{A combination of first using UAVs and the deploying directed UGVs will permit the DIM sequence to be employed. This is turn will reduce the complexity and eventual cost of detection and identification.}".
\end{itemize}

The first research question is based on these identified gaps and is explored in Chapters \ref{chapter:SceneSurveying} and \ref{chap:targetLocalisation}. The second question arose from the first research question, since prototyping and validating systems to be run on RAVs in hazardous environments in the real world is prohibitive due to the high cost.