This rest of this thesis is structured as follows: 
\begin{enumerate}
    \item Chapter \ref{chapter:Background} outlines some essential background knowledge related to the work discussed in this thesis. We provide sections on the ROCSAFE project, agent and multi-agent systems, Hidden Markov Models, Dynamic Bayesian Networks, sequential statistical Hypothesis Testing and high-fidelity game engine environments. We also provide a review of related work.
    
    \item Chapter \ref{chap:HighFidelitySim} provides the details of the development of the high-fidelity simulation environment mentioned above in the contributions section. We provide an exposition on the techniques that we used to achieve a high level of realism. We also outline the integration of a plugin called Airsim \cite{Shah2017AirSim:Vehicles}, which facilitated the use of virtual \textbf{R}emote \textbf{A}erial \textbf{V}ehicles (RAVs). We finally discuss potential future work and applications for the simulation environment.
    
    \item Chapter \ref{chapter:SceneSurveying} is concerned with outlining the solution we implemented to solve the problem of Scene Surveying and is concerned with the first research question stated in Section \ref{sec:ResearchQuestions}. A simplified version of the problem is first explored. We provide algorithms and some sample results outlining the advantages of the explored solution. We then move to the complete version of the problem, that introduces additional constraints. We discuss our implemented solution, which uses an open-source constraint-programming solver.
    
    \item Chapter \ref{chap:targetLocalisation} addresses the target localisation problem using a system of RAVs. This pertains to the first  research question stated in Section \ref{sec:ResearchQuestions}. We provide the details of the agent-based approach that we took to solve this problem. We explain how a Dynamic Bayesian Network (DBN) was used to describe the conditional relationships between relevant stochastic features of the agent's environment. Results of running simulations while perturbing user-specified parameters are provided in order to show how some configurations can lead to much more desirable results than others, which offer insight into how to calibrate the search parameters for the specific application of a potential user.
    
\end{enumerate}