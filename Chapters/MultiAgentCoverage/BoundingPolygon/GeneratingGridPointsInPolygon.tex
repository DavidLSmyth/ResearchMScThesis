\subsection{Generation of Grid Points}
note{From C&B Analysis of sequential decision-making using prob. search. Consider a bounded discretized search areaA , which is de-fined byC disjoint cells. This discrete representation can charac-terize numerous environment types of diverse spatial scale, such as open areas that are relevant to maritime search operations, cluttered regions, such as obstacle-filled arenas, or structured environments, such as rooms and hallways in a building. Other factors, which include the geometry and extent of the searcher’s sensor footprint, and the size of the sought object, or other op-erational considerations (e.g., existing coordinates or reference systems) can also govern the specific cellular decomposition of the search area}

\note{More general sensor models that account for additional spatial and/or temporal dependences be-cause  of  clutter  (indoor)  or  terrain  and  atmosphere  (outdoor) can  be  constructed  (e.g.,αs(k),kandβs(k),k )  but  is  deferred for future study}


\note{In other words, the greater hindrance to deciding that a target is present in the search cell is the false-positive detection probabil-ity, since false alarms tend to prevent the searcher from “trust-ing” its positive observation. In contrast, if the missed detection probability is high, then the searcher cannot declare the search cell empty of the target with high confidence without expending multiple observations in the c}







