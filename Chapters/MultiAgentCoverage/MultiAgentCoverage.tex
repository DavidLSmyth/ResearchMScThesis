%!TEX root = ../thesis.tex
%*******************************************************************************
%****************************** Third Chapter **********************************
%*******************************************************************************
\chapter{Multi-Agent Coverage Problem}
\note{Maybe rename this chapter}
In this chapter, we provide details of the techniques and software developed to solve the problem <need to think about this>.

As with the rest of the work in this thesis, this chapter is motivated by the ROCSAFE project. In the initial phase of any crime scene investigation, it is desirable for the crime scene manager to scan the entire crime scene area to thoroughly assess the scene as early as possible \cite{TechnicalWorkingGrouponCrimeSceneInvestigation2013CrimeEnforcement}. We will henceforth refer to the area to be scanned as the \textit{region of interest} for the purpose of generality. We address this problem using a fleet of UAVs, which are equipped with cameras and sensors which are highly suited to data-gathering. The information acquired by this initial sweep can be used for a multitude of purposes and the sooner it can be gathered, the better, since it will often be subjected to time-consuming analysis. We have attempted to treat this problem so that the solutions explored can be applied to other domain-specific contexts. The problem naturally breaks down into smaller sub-problems:
\begin{itemize}
    \item Given a polygonal region defined by a set of points that lie on the earth's surface, generate a discrete set of cells of uniform area that partition the region 
    %according to a user-specified height and width. 
    The cells can be described only by their center points, as they are assumed to be of uniform height and width. We will refer to this set as $R$, the set of \textit{waypoints}.
    \item Find a set of routes for each of $K$ UAVs to be used in the data-gathering process, such that these sets partition $R$ and the cost of the system of UAVs traversing these points is minimized. Cost functions of the traversal are discussed in <reference relevant section>.
\end{itemize}

Hence there are two main challenges: find a discrete representation of a continuous area which represents the region of interest, and then find routes that the UAVs can execute in order to visit each of the grid points exactly once, while minimising the cost of doing so.


\section{Generation of Waypoints}
We first address the issue of finding a discrete set of waypoints, referenced above as $R$. We chose to formulate the problem using a discrete set of waypoints, rather than deal with the case of treating the region as continuous space, since it is much easier to accurately record sensor data at a given point. Discretising the space also allows the problem to be formulated in a manner that is easier to tackle, without losing the key aspects.
%\begin{itemize}
    %\item It is more straightforward to describe the problem using a discrete set of points rather than a continuous one.
%    \item It is much easier to process sensor information if it is known exactly where this data has been captured. The UAVs may stop at each discrete waypoint to visit these locations. *Focus on this point, expand if possible*
%    \item The case of path-finding using discrete waypoints has a strong body of literature behind it
%    \item The UAVs run autopilot software which can guide them using the on-board GPS sensor
%\end{itemize}

%which are the center points of cells that partition the region of interest.
\note{set of waypoints create a voronoi partition, might be worth talking about}

\note{Above needs revision, will come back once rest of chapter has been fleshed out a bit more}
From a practical perspective, the region specified is defined over the earth's surface, which is not planar. The generation of grid points takes this into account. The earth can be well described mathematically as an ellipsoid talk about WGS84






