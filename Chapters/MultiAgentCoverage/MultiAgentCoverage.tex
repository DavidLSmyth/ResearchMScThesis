%!TEX root = ../thesis.tex
%*******************************************************************************
%****************************** Third Chapter **********************************
%*******************************************************************************
\chapter{Scene Surveying}\label{chapter:SceneSurveying}
In this chapter, we provide details of the techniques and software developed to address the problem of aiding autonomous scene surveying, which was identified as part of the first research question stated in Section \ref{sec:ResearchQuestions}: 
\\
\\
"\textit{Can an agent-based software system be developed to run on a system of heterogeneous autonomous aerial vehicles to aid the tasks of scene surveying and target localisation in a hazardous environment? }".
\\
\par As with the rest of the work in this thesis, this chapter is motivated by the context of the ROCSAFE project, but we present the work in a general setting. In the initial phase of any crime scene investigation, it is desirable for the crime scene manager to visually scan the entire crime scene area to thoroughly assess the scene as early as possible \cite{TechnicalWorkingGrouponCrimeSceneInvestigation2013CrimeEnforcement}. We will henceforth refer to the area to be scanned as the \textit{region of interest} for the purpose of generality. In the ROCSAFE project, it is intended to address this problem using a fleet of RAVs, which are equipped with cameras and sensors which are highly suited to data-gathering. The RAVs can operate concurrently and can communicate with a centralised controller. We developed a system to autonomously survey a hazardous situation using the simulation environment described in Chapter \ref{chap:HighFidelitySim}. We published this work in a number of papers \cite{Smyth2018AInvestigation}, \cite{Smyth2018ASupport}, \cite{Smyth2018UsingDrones}.
%The information acquired by this initial sweep can be used for a multitude of purposes and generally speaking the sooner it can be gathered, the better, since it will often be subjected to time-consuming analysis. 
%We used the simulation environment described in Section \ref{chap:HighFidelitySim} to test the developed system.
%In order to ensure that the solutions explored are not domain-specific, we make few assumptions regarding the nature of the hardware used to solve the problem. 
\par We began by providing a description of the two main sub-problems identified as part of autonomous scene-surveying:
\begin{enumerate}
    \item Given a polygonal region defined by a set of points that lie on the earth's surface, generate a discrete set of cells of uniform area that partition the region of interest.
    %according to a user-specified height and width. 
    The cells can be described by their center points, as they are assumed to be of uniform height and width. We will refer to this set as $R$, the set of \textit{waypoints}.
    \item Find a set of routes for each of $K$ RAVs to be used in the data-gathering process, such that these sets partition $R$ and the cost of the system of RAVs traversing these points is minimized. %Objective functions which define the cost of traversal are discussed in <reference relevant section>.
\end{enumerate}

Hence there are two main goals: find a discrete representation of a continuous area which represents the region of interest, and then find routes that the RAVs can execute in order to visit each of the grid points exactly once, while minimising the cost of doing so. 
%We then give a mathematical definition of the sub-problems in sections \ref{sec:GenerationOfWaypoints}, \ref{subsec:SimplifiedVRP} and \ref{sec:SceneSurveyingBatteryConstraints}.


\section{Generation of Waypoints}
We first address the issue of finding a discrete set of waypoints, referenced above as $R$. We chose to formulate the problem using a discrete set of waypoints, rather than deal with the case of treating the region as continuous space, since it is much easier to accurately record sensor data at a given point. Discretising the space also allows the problem to be formulated in a manner that is easier to tackle, without losing the key aspects.
%\begin{itemize}
    %\item It is more straightforward to describe the problem using a discrete set of points rather than a continuous one.
%    \item It is much easier to process sensor information if it is known exactly where this data has been captured. The UAVs may stop at each discrete waypoint to visit these locations. *Focus on this point, expand if possible*
%    \item The case of path-finding using discrete waypoints has a strong body of literature behind it
%    \item The UAVs run autopilot software which can guide them using the on-board GPS sensor
%\end{itemize}

%which are the center points of cells that partition the region of interest.
\note{set of waypoints create a voronoi partition, might be worth talking about}

\note{Above needs revision, will come back once rest of chapter has been fleshed out a bit more}
From a practical perspective, the region specified is defined over the earth's surface, which is not planar. The generation of grid points takes this into account. The earth can be well described mathematically as an ellipsoid talk about WGS84










\section{Scene Surveying}
This section outlines the algorithms that were explored to generate a set of routes for the UAVs that will be used to record sensor data at each of the grid points generated in the region of interest, which are generated using the algorithm described in the preceding section. We make some assumptions related to the solution of this problem:
\begin{itemize}
    \item Each UAV is assumed to have the same internal representation of the region of interest, namely the set of grid points generated by algorithm \ref{alg:GridGeneration}.
\end{itemize}

Talk about how problem was transformed to TSP problem, took that and then divided up soln for mTSP.

\subsection{Nearest-Neighbor algo}
Outline the NN algo in "Using a Game Engine to Simulate Critical Incidents
and Data Collection by Autonomous Drones"

\subsubsection{}

\subsection{Executing Agent Routes in Simulation}
\note{talk about how routes can be executed in sim. env.}
In order to test the algorithms developed in this chapter in a realistic scenario, we utilised the simulation environment described in Chapter \ref{chap:HighFidelitySim}. The AirSim plugin \cite{Shah2017AirSim:Vehicles} provides a straightforward interface to be able to configure the simulation environment to run with a user specified number of UAVs, with an API to send commands to each. We successfully ran a number of tests in simulation using the User Interface to select the bounding area and grid spacing for the UAVs, which interfaced with the code which generated the routes for the UAVs. We then used the AirSim API to send the virtual UAVs to the generated waypoints on each route to gather images. The results of a simulation run showing the executed routes are shown in figure \ref{fig:VirtualPlannedRoutes}


\begin{figure}
    \centering
    \includegraphics[width=0.6\textwidth]{Chapters/SimulationEnv/Figs/DebuggingLines/RoutesWithRAVsVisible.png}
    \caption{UAVs executing their planned routes in the simulation environment}
    \label{fig:VirtualPlannedRoutes}
\end{figure}

\section{Scene Surveying With Heterogeneous Battery Constraints and Sampling Times}
\note{Keep this very brief - there is not a great research output here. Simply mention that we began looking into using solvers to calculate solutions.}

Once we had prototyped the NN algorithm in order to find a suitable solution to the simplified problem, we then considered the more general problem, where UAVs have heterogeneous battery constraints, sampling times and operational speeds, which is mentioned in section \ref{sec:SceneSurveying}. This addresses the more general Vehicle Routing Problem (VRP). We decided to formulate the problem as a \textit{linear program} and planned to find solutions using a linear program solver, following the literature outlined in section <reference>. This means that the problem's objective function and constraints must be written as linear expressions.
%A number of code repositories with permissive licences that provide solvers for linear programs exist .S
Specifying problem explicitly as a linear program and then passing it to a solver is a time-consuming process, which has led to a number of tools developed which act as wrappers for developers to solve well-known problems without having to write excessive amounts of boiler-plate code. We chose to use Google's Apache-licensed \href{https://developers.google.com/optimization/}{\textit{Operations Research}}\footnote{\href {https://developers.google.com/optimization/}{https://developers.google.com/optimization/}} (OR) repository, which contains a routing library with high-level interfaces specifically designed to allow the user to define and solve VRPs. This meant we could focus on defining the salient aspects of the problem rather than the details of how to convert the VRP into a linear program.

We began by following the example outlined in the \href{https://developers.google.com/optimization/routing/vrp}{OR tools documentation}\footnote{\href {https://developers.google.com/optimization/routing/vrp}{https://developers.google.com/optimization/routing/vrp}}. The documentation outlines the steps to solve a linear program using the repo follow the same pattern: 
\begin{itemize}
\item Create the variables.
\item Define the constraints.
\item Define the objective function.
\item Declare the solver — the method that implements an algorithm for finding the optimal solution.
\item Invoke the solver and display the results.
\end{itemize}

In order to follow these steps, we added the following to the provided example code:

\subsection{Specifying Variables and Constraints}
We first implemented a vehicle class which records the variables related to routing for each vehicle. Member variables included:
\begin{itemize}
    \item The time taken to record data at each node
    \item The location of the depot, where the charging point is located
    \item The operational speed of the vehicle
    \item The estimated remaining battery life in seconds
\end{itemize}
In order to facilitate recharging, we created virtual depot nodes, which are duplicate nodes of the recharge location and have an associated recharge time for each UAV.

For each UAV, we then created a matrix of times taken to travel from one node to the next and service the second node, based on the distances between the nodes, the operational speed of the UAV and the service time.

In order to specify the objective function, it is necessary to monitor some quantities that accumulate along a vehicle's route. This is done using the OR Tools interface by adding a \textit{dimension}. Dimensions are used to track each vehicle's cumulative travel time using the matrix of transition times mentioned above. This specifies in the objective function the cost of traversing arcs for each UAV. We force the UAVs to visit a recharge depot once its battery level reaches below 10\% by specifying a cumulative variable with a range equal to the predicted time taken for the RAV battery to degrade to this level. The solver must include the dummy depot node in the solution at this time. By adding the dimensions that track the amount of time it takes vehicles to traverse and gather data at each node, along with mandatory visits to the dummy nodes at the charging station..

We specified each time dimension to contribute to the objective function by using the setGlobalSpanCostCoefficient method, which notifies the solver to minimise the largest cost among all the time dimensions, which equates to minimising the longest time taken for any vehicle to complete its route.

We used the default solver to calculate a solution, with the cheapest arc heuristic as the 



\subsection{Specifying a Time Dimension}

\subsection{}

\subsection{}


\begin{itemize}
    \item The OR Tools interface requires the user to specify either a distance matrix or a callback method which calculates the cost of traversing an arc between nodes. We used the Haversine distance to create a callback method to calculate the distances between points.
    \item We wrote a time evaluator method, which 
\end{itemize}





\section{Conclusion and Future Work}\label{sec:SurveyingConclusionFutureWork}
In this chapter we provided the details of how a swarm of RAVs can be used to gather data in a given region described by a polygon. We first described how to discretise the region into a set of grid points. We made some minor modifications to ensure that the implementation would be able to deal with real-world usage, which in practice takes the form of using the coordinate system that GPS depends on. We then implemented a Nearest Neighbor (NN) heuristic algorithm to generate routes for the swarm of RAVs which visit each grid point in the region exactly once, which is a solution to the multiple Travelling Salesman Problem. For practical purposes in relation to ROCSAFE \cite{Bagherzadeh2017ROCSAFE:Incidents}, the project that has funded this work, this offered a simple and scalable solution to the surveying problem. In order to run tests, we then used a web development framework to design a UI to interact with the grid generation code and the NN algorithm, which we used to generate some results which can be seen in Tables \ref{table:NNAlgoResultsRect}, \ref{table:NNAlgoResultsTri}, \ref{table:NNAlgoResultsHex}. These experiment were purposely set up to demonstrate how regions of interest that have symmetries allow the NN algorithm to find solutions which scale well to multiple RAVs. %\ref{table:NNAlgoResultsIrregular}.
Finally, we used the simulation environment, outlined in Chapter \ref{chap:HighFidelitySim}, to run further tests which offer highly realistic conditions, since the virtual RAVs run autopilot software that is used on real-world RAVs. These tests were carried out successfully and the results are illustrated in Figure \ref{fig:VirtualPlannedRoutes}. 

We then began investigating using an optimisation tool to solve the more general problem of surveying with constrains such as heterogeneous RAV battery capacities, sampling times and operational speeds. We used the open-source Apache licensed Google OR-Tools repository to develop a prototype solution. This solution took orders of magnitude longer to calculate solutions compared to the NN algorithm, but offered qualitatively good solutions to the more general problem.



\subsection{Future Work}
We outline some future work that we planned but did not have the time to explore in depth. The points outlined are all taken to be specific to the Euclidean grid-based approach that was discussed throughout this chapter.

\begin{enumerate}

    \item A major limitation to the work done in this chapter is that we assume that RAVs act deterministically. In reality, there are many sources of randomness in environments that RAVs operate in. An approach that can deal with this randomness would have a major advantage over any approach that assumes determinism.
    
    \item Our approach is centralised, which means that RAVs do not act as independent agents. If for some reason an agent failed or communications were lost between RAVs and the central controller, the system would fail. Future work could focus on modifying the approach taken in this thesis which could allow the RAVs to re-negotiate on what work they do based on their capabilities. Re-negotiation could periodically occur in the case that a RAV is left idle when it could be participating in the task.
    
    %\item The quality of the solution to the TSP given by the NN algorithm is known to be dependent on the start position <suitable reference missing>. A common improvement to the algorithm is to run it $n$ times for each possible starting point at each node in the network of $n$ nodes, and then take the best solution found. In the case of the mTSP, for $k$ RAVs, this would mean exploring ${n \choose k}$ different combinations of starting positions, which may make the running time of the algorithm prohibitively long. Therefore, future research could explore an effective way of determining the best starting nodes for the RAVs without implementing a brute force search to improve the NN solution.
    %\item Many solutions to the mTSP problem use \textit{iterative improvement} strategies,  which use an existing algorithm to create a solution, then have another algorithm improve the solution.\textit{ Ensemble} strategies are also frequently used, which applies a set of algorithms to the  problem and returns the best solution. The use of either of these techniques with the NN algorithm could provide significantly better results, especially given that this has been shown in the case of Euclidean distance cost functions. For example, reversing some segments that "cross" guarantees an improvement via the \textit{triangle inequality}, as shown in Figure \ref{fig:fixingTourCrossing}. 
    %\item Many open-source and commercial linear program solvers exist which are optimized to deal with the mTSP and VRP. We discussed how we used Google's OR-Tools repository <link in footnote> to prototype a solution. Future work could involve optimising the configuration of the problem and choice of back-end solver to produce a scalable solution.
    %\item Representing the problem in a different manner may yield a suitable solution that is easier to computer. For example, representing the region to survey as continuous space rather than a discrete grid may facilitate the development of a control algorithm that is simpler and more effective in solving the surveying problem than the ones that exist that deal with discrete grids.
    \item Theoretical analysis of the performing surveying tasks with a swarm of RAVs may yield some insight into how to create a scalable algorithm for certain classes of grid (e.g. some regular shapes or compositions of regular shapes).
    \item The surveying task can be used to provide data to other algorithms that can provide useful tools when managing a disaster scene. For example, \textit{structure-from-motion} is a technique for creating three-dimensional point clouds from two-dimensional images. A review of techniques used for structure-from-motion can be found in \cite{Bianco2018EvaluatingPipelines}. There may be potential to perform optimisations in the generation of the spacing and altitude of the grid points as well as the sequence in which the points are collected in order to generate the most accurate point cloud possible. We use the sample data collected as prior information for the stochastic target localisation problem outlined in Chapter \ref{chap:targetLocalisation}
    
    
\end{enumerate}


%\begin{figure}
%\centering
%\begin{minipage}{.5\textwidth}
%  \centering
%  \includegraphics[width=.48\linewidth]{Chapters/MultiAgentCoverage/Figs/crossingSegments.PNG}
%  \captionof*{figure}{"X" present in a tour}
%  \label{fig:crossingTour}
%\end{minipage}%
%\begin{minipage}{.5\textwidth}
%  \centering
%  \includegraphics[width=.48\linewidth]{Chapters/MultiAgentCoverage/Figs/nonCrossingSegments.png}
%  \captionof*{figure}{"X" removed from tour by reversing segments}
%  \label{fig:nonCrossingTour}
%\end{minipage}
%\caption{Using the triangle inequality to improve tours.}
%Figures based on those found in \href{https://github.com/norvig/pytudes/blob/master/ipynb/TSP.ipynb}{Norvig's TSP github repo.}}
%\footnote{\href {https://github.com/norvig/pytudes/blob/master/ipynb/TSP.ipynb}{https://github.com/norvig/pytudes/blob/master/ipynb/TSP.ipynb}}
%\label{fig:fixingTourCrossing}
%\end{figure}
%\footnote{\href {https://github.com/norvig/pytudes/blob/master/ipynb/TSP.ipynb}{https://github.com/norvig/pytudes/blob/master/ipynb/TSP.ipynb}}























