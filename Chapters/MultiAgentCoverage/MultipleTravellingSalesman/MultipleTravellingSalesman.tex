\nomenclature[]{VRP}{Vehicle Routing Problem}

\section{Scene Surveying}
This section outlines the algorithms that were explored to generate a set of routes for the UAVs that will be used to record sensor data at each of the grid points generated in the region of interest, which are generated using Algorithm \ref{alg:GridGeneration} described in the preceding section. We make some assumptions related to the solution of this problem:
\note{This list might not cover everything, come back to it}
\begin{itemize}
    \item Each UAV is assumed to have the same internal representation of the region of interest, namely the set of uniformly spaced grid points generated by Algorithm \ref{alg:GridGeneration}.
    \item Each UAV is assumed to have the ability to move between any pair of grid points unobstructed using the shortest possible path.
    \item UAVs are assumed to move with a fixed operational velocity, which can vary between UAVs.
    \item Each UAV may be equipped with different sensors to the others and it is assumed that sensing times may varying among the UAVs.
    \item Each UAV has a finite battery capacity which implies they have a finite amount of time that they can fly for before they need to recharge.
\end{itemize}

%Talk about how problem was transformed to TSP problem, took that and then divided up soln for mTSP.

The scene surveying problem that we would like to solve can be treated as an instance of the well-known \textit{Vehicle Routing Problem} (VRP), which first appeared in a paper by \citeauthor{Dantzig1959TheProblem} \cite{Dantzig1959TheProblem}. This problem is a generalisation of the classic \textit{Travelling Salesman problem} (TSP). There are many variants and extensions to this problem, but it essentially asks for a set of routes to be assigned to the UAVs such that each point in the network is visited exactly once, which minimises the total time taken to "service" each of the points. In our case, the service time is the time taken to record a sensor reading. A formal definition of the VRP and its variants can be found in \cite{Toth2002TheProblem}.

\subsection{Simplified Problem}
\note{maybe change this to constrained problem or something similar}
We began by taking a simplified version of the full vehicle routing problem in order to explore possible solutions. Here we expand on our published work in \cite{Smyth2018UsingDrones}, which summarises how this simplified problem was tackled. Rather than concern ourselves with the details of sensor sampling times and battery constraints, we first focused on designing a solution that can assign a set of routes to a homogeneous set of UAVs. 
%We assumed that service times add a fixed constant to the total time taken to perform the survey (and can hence be ignored) and we ignore the time added that the UAVs might need to recharge. 
The problem can be described as follows:
\\
\textit{Given a fully connected graph, $G$, to visit and $n$ RAV agents, find a subtour for each agent such that each point in $P \in G$ is visited exactly once by any agent in the system, with the objective of minimizing the longest time taken for any individual agent subtour, in order to minimize the time taken to carry out the survey.}
\\
This is simply the \textit{multiple travelling salesman problem} (mTSP). 

%According to the number of distribution centers: single distribution center and multi-distribution center problem;
%According to the type of vehicle: single-vehicle type and multi-vehicle type problem;
%According to the characteristics of the task: pure send (take) cargo problems and loading and unloading mixing problem;
%According to whether the time constraints: no time window problem and time window problem;
%By vehicle loading: And the problem of non-full load;
%According to the optimization of the number of goals: a single objective and multi-objective problem;
%Vehicle and vehicle by the ownership of the points: the vehicle open problem and vehicle closure problems;
%By mastering the information of certainty: Sexual VRP and non-deterministic VRP problems;
%As can be seen from these classifications, solutions to the VRP problem are varied, each category 

\subsubsection{Explored Solutions}
First we show that the mTSP is at least as hard as the TSP, which implies that finding a polynomial time solution is not feasible. We then discuss solutions that were explored that give non-exact solutions. 

\\Proof:\\
Let T be a tour that is an optimal solution to the Travelling Salesperson Problem (TSP) for  a graph G(V, E), where E $\subseteq$ V $\times$ V, with a cost function $c(p_i, p_j)$ defined for all $(p_i, p_j)$ $\in$ E and an induced cost function 
C(T) = $\sum\limits_{(p_i, p_j)\in S_k}C(p_i, p_j)$ defined for any tour of G.
The tour T is an ordered tuple (($p_i, p_j$), ($p_j, p_k$),..., ($p_m, p_n$)) $\subseteq$ E which satisfies Cost(T) = \(\displaystyle \min_{e \subseteq E}\) $\sum_{(p_i, p_j) \in e}$ c($p_i, p_j$) with the constraint that each $p_i \in$ V must be visited exactly once. This means T is a hamiltonian tour of G of minimal cost. Any partition ($T_1, T_2, ..., T_m$) of a solution T' can be written as C(T')=
\\

$\sum\limits_{k=1}^{m}\sum\limits_{(p_i, p_j)\in T_k} c(p_i, p_j) \leq m\times$
\(\displaystyle\max_{T_k \in T}\)
$\sum\limits_{(p_i, p_j)\in T_k}c(p_i, p_j)$
\\

\noindent and since Cost(T) $\leq$ Cost(T'), for the partition determined by Algorithm \ref{alg:agentRoutesEdited}, we find that for any solution S to the Multiple Traveling Salesperson (MTS) problem consisting of the partition ($S_1, S_2, ...,S_m$) for each of the m agents:\\
Cost(T)
$\leq$ Cost(S) $\leq$  m$\times$
(\(\displaystyle\max_{S_k \in S}\)
$\sum\limits_{(p_i, p_j)\in S_k}C(p_i, p_j)$) = Cost of MTS solution, S.
This means we can use a solution of the standard Traveling Salesperson problem as a lower bound for comparison with Algorithm \ref{alg:agentRoutesEdited}. For example, the Held-Karp algorithm gives a well-known lower bound when dealing with a metric space \cite{VALENZUELA1997157}.

We explored a number of solutions to this 

\subsubsection{}
Outline the NN algo in "Using a Game Engine to Simulate Critical Incidents
and Data Collection by Autonomous Drones"

\subsubsection{}