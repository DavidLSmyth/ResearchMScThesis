\nomenclature[]{VRP}{Vehicle Routing Problem}

\section{Scene Surveying}
This section outlines the algorithms that were explored to generate a set of routes for the UAVs that will be used to record sensor data at each of the grid points generated in the region of interest, which are generated using Algorithm \ref{alg:GridGeneration} described in the preceding section. We make some assumptions related to the solution of this problem:
\note{This list might not cover everything, come back to it}
\begin{itemize}
    \item Each UAV is assumed to have the same internal representation of the region of interest, namely the set of grid points generated by algorithm \ref{alg:GridGeneration}.
    \item Each UAV is assumed to have the ability to move between any pair of grid points unobstructed using the shortest possible path.
    \item UAVs are assumed to move with a fixed \textit{operational velocity}, which can vary between UAVs.
\end{itemize}

%Talk about how problem was transformed to TSP problem, took that and then divided up soln for mTSP.

The scene surveying problem that we would like to solve can be treated as an instance of the well-known \textit{Vehicle Routing Problem} (VRP), which first appeared in a paper by \citeauthor{Dantzig1959TheProblem} \cite{Dantzig1959TheProblem}. This problem is a generalisation of the classic \textit{Travelling Salesman} problem. It asks "What is the optimal set of routes for a fleet of vehicles to traverse in order to deliver to a given set of customers?"

\subsection{Nearest-Neighbor algo}
Outline the NN algo in "Using a Game Engine to Simulate Critical Incidents
and Data Collection by Autonomous Drones"

\subsubsection{}