\subsection{Initial Assumptions}\label{subsec:initalAssumptions}
\note{may need to rename this. want to convey that initially, we made some simplifying assumptions that isolate key aspects of problem that need to be solved. Then these assumptions were modified to deal with the more complex problem involving battery etc.}

Rather than immediately attempting to tackle the full problem, we chose to initially make some simplifications in order to identify potential solution strategies that could be extended to more complex versions of the problem. At the outset, we made the following simplifying assumptions:
%As outlined in the literature review, this problem has been approached before by treating the problem as a 2 Time Slice Dynamic Bayesian Network (2TDBN). 
\begin{itemize}
    \item There are either zero or one targets to be localized.
    \item There is only a single RAV operating in the region.
    \item The RAV has unlimited battery capacity.
    \item The region to be searched can be well approximated by a polygon.
    \item The sensor specificity and sensitivity are known or can be estimated for a given resolution (e.g. 1m). These are assumed to be greater than 50\% for the given resolution.
    \item The search agent(s) operate over a discrete spatial grid spanning the region to search, assumed to be polygonal as above, the dimensions of which are pre-determined by the sensor resolution.
    \item The RAV is assumed to have a GPS sensor that is accurate to beyond the sensor resolution (implying that the RAV moves to discrete grid locations without drift).
    \item The target is assumed to be small enough to occupy only one grid cell at a time. It is also assumed to not lie across grid cells.
    \item There exists a path that the RAV can follow between any two given grid cells.
\end{itemize}
While these assumptions are clearly unrealistic, they are convenient because they simplify the design of the system and subsequent analysis. In later sections in this chapter, these assumptions are relaxed and the necessary modifications for the solution strategy are discussed. Some ramifications of these assumptions are addressed later in the chapter, at section <x>. It is worth noting that similar simplifying assumptions were made in related works in the literature, 
(\cite{Chung2007ASearch} and \cite{Waharte2010SupportingRAVs}), % find additional citations in mendeley
which strongly influenced our initial approach.