The next assumption that we address is that only a single agent will be used to execute the search. Using a single agent which has complex mechanisms for maneuvering and sensing poses a significant risk due to the very real chance of catastrophic malfunction. For this reason, much recent research been focused on multi-agent systems which can offer robustness, redundancy and scalability where single-agent system can't \cite{Stone2000MultiagentPerspective}. In the context of the target localisation problem, it is intuitive to conclude that the lower bound of the search time is inversely proportional to the number of agents carrying out the search, which is further motivation to include mechanisms for using multiple agents for carrying out the search.

The ROCSAFE project \cite{BagherzadehROCSAFE:Incidents} motivating this work assumes that the RAVs used to carry out the search procedure may lose connectivity with a centralised controller from time to time, which means that the assumption that they must be able to act autonomously without a centralised decision mechanism must be made. There are many well-researched paradigms that multi-agent systems fall into, which allow high-level goals to be achieved without relying on a centralised source of control. Due to time constraints, we did not investigate deeply negotiation strategies between agents or ways of inducing emergent system behaviours, but this could be a topic of future research for this area.\par
%negotiation strategies may be employed to achieve a high-level goal \cite{Kraus1997NegotiationEnvironments}, a pre-defined strategy maybe be followed, or adaptive emergent behaviours may 
We assumed that the agents can move to any unoccupied grid cell within a neighborhood of their current location for each time step of the search. The sensing systems in the ROCSAFE project take a significant amount of time to analyse a sample, whereas the RAVs are highly mobile and moving between grid locations takes much less time than sensing. This is a minor detail and can be easily modified to reflect a relatively small sensing time. We also assume that the RAVs may communicate pair-wise with each other within a certain radius, with the probability of a successful transmission inversely proportional to the distance between the RAVs. We also allow that the sensors on the RAVs may have different probabilities of reporting false positives and false negatives, which reflects that some senors used may be more accurate than others, due to cost, size and other realistic constraints. \par

Since we do not have an associated cost with communication, we implemented a strategy whereby agents will attempt to communicate with all other agents if possible, after they have sensed at each timestep. As in \cite{Waharte2009CoordinatedRAVs}, each agent keeps a local list of sensor readings made by itself and other agents. When agents communicate, the agent updates its local state estimate using algorithm \ref{alg:bayes_filter_observations_only} with the previously unseen sensor readings communicated by the other agent and the calibrated false positive and false negative rate for the other agent's sensor. This takes advantage of the fact that the Bayesian update rule for the estimated state of each agent is composed of multiplicative terms, for which the order of multiplication does not matter. Information related to the locations of targets will naturally propagate around the system as agents come within the communication radius of each other. Future research could incorporate a search strategy that optimises the trade-off between agents exploring independently and gathering new information against exploiting the information that other agents have already gathered by attempting to sychronise their search paths to remain within the communication radius.