\subsection{Action Selection Strategies}\label{subsubsec:ActionSelection}
At each time step the agent may either choose to terminate the search or move to a new location to take a measurement reading. The agent may choose which of these actions to perform based on its location, the search status and its internal representation of the environment, which are outlined in the preceding section. First we discuss how move actions are chosen, if the agent decides that the search should not be terminated at the current time step. We began by implementing some of the recommended basic search strategies outlined in the related works by \citeauthor{Chung2007ASearchb} \cite{Chung2007ASearchb} and \citeauthor{Waharte2010SupportingUAVsb} \cite{Waharte2010SupportingUAVsb}. These search methods are devised in order to optimize the agent's performance measure, which is set out in \ref{sssection:PerfMeas}. The results of applying these strategies are discussed in \ref{sec:SimulationResults}.\par 
\note{In the following, might be good to mention the motivation in relation to the performance measure}
\textbf{Random Search}
This method serves as a baseline against which similar strategies can be compared. The agent simply chooses the next grid location to explore randomly from all possible grid locations. The expected number of moves needed to take a reading at all possible grid cells is given by the solution to the\textit{coupon collectors problem} \cite{Erdos1961OnTheory}, $nH_n$, where $H_n$ is the $n_{th}$ harmonic number and $n$ is the total number of grid cells, which gives a good idea of the expected amount of time that will be taken to conclude the search.

\textbf{Greedy Search}
A simple greedy search was implemented next, where the agent chooses to visit the grid cell with the highest estimated probability of containing the target in a localized region around the agent:
\footnotesize
\[
Action_t = \argmax_{NewLoc \in N(AgentLoc)}{p(TargetLoc = NewLoc, SearchStatus, AgentLocation| e_{1:t}, u_{1:t})}
\]
\normalsize
$N(AgentLoc)$ is a function that returns a neighborhood of locations around the agent's location and can be calibrated to trade off the cost of saccading between grid cells that may be far from each other against the cost of limiting the agents range to a narrow and possible "cold" region.This method is designed bearing in mind that there may be motivation to explore some areas before others based on prior knowledge. 

%\textbf{$\epsilon$-greedy Search} Greedy search  simplest method to implement was an $\epsilon$-greedy action selection method, whereby the agent moves to the grid cell with the highest estimated probability in a pre-defined radius with probability 1-$\epsilon$ and a random grid cell in the pre-defined with probability $\epsilon$. This encourages the agent to exploit it's current know\par

\textbf{Sweep Search}
This strategy sweeps the region of interest systematically, aiming to take a reading at each grid cell an equal number of times while minimizing the total distance travelled. It then traverses this same path again in reverse order. This required planning a trajectory in advance, which is often referred to in related literature as the \textit{complete coverage path}. Since the region to be swept is assumed to be a grid, where adjacent points are assumed to be equidistant, a number of heuristic solutions were available for use from Section \ref{sec:SceneSurveying}.\par

%\textbf{POMDP Based Search} 
%\note{Not sure if I should go into this at all. Work was done in understanding POMDPs and clearly the solution can be easily framed as a POMDP but my work stopped short in actually finding a solution to POMDP that works well}
%As outlined in 
%details. \par