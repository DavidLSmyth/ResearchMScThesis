\subsection{Actuators and Sensors}
Here we consider the actions that may be chosen to be performed by actuators and percepts that may be received by sensors. The problem of \textit{target localization} in the context of this chapter requires the agent to move around a discrete grid and use a calibrated sensor to record noisy readings that indicate whether the target is present or not at the location of the reading. It is therefore intuitive to describe the set of possible actions to be performed by the actuators by the set of all $n$ possible grid locations that the agent can move to and take a sensor reading at, indexed by an arbitrary ordering: $\{move\_x_1, move\_x_2, ..., move\_x_n\}$. We add additional actions to this set, $\{terminate\_search\_x_{i}\}$, for $i \in \{1, 2, ..., n, n+1\}$, which lead to an absorbing terminal state representing the agent's conclusion regarding whether a target is present or not, $terminated\_x_{i}$ . The set of percepts that the agent will receive from its sensors come from the binary set \{1, 0\}, indicating the target has or has not been detected, respectively.