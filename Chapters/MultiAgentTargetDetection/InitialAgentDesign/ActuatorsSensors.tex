\subsection{Actuators and Sensors}
Here we consider the actions that may be chosen to be performed by actuators and percepts that may be received by sensors. The problem of \textit{target localisation} in the context of this chapter requires the agent to move around a discrete grid and use a calibrated sensor to record noisy readings that indicate whether the target is present or not at the location of the reading. We therefore describe the set of possible actions to be performed by the actuators by the set of all $n$ possible grid locations that the agent can move to and take a sensor reading at, indexed by an arbitrary ordering: $\{move\_x_1, move\_x_2, ..., move\_x_n\}$. We did not restrict the agent to only move between adjacent grid cells since our use case deals with agile aerial vehicles, which can move freely between any two grid points. We add search termination actions to this set, $\{terminate\_search\_x_{i}\}$, for $i \in \{1, 2, ..., n, n+1\}$, which lead to an absorbing terminal state representing the agent's conclusion regarding whether a target is present or not, $terminated\_x_{i}$. To summarise, the agent may either choose a move action or a search termination action, which respectively move the agent to a new grid location at which record a sensor reading, or conclude the search and return the most likely target location $x_i$. \par

The set of percepts that the agent will receive from its sensors come from the binary set \{1, 0\}, indicating the target has or has not been detected, respectively.\par
