\subsection{Agent Environment}
\note{Use of italics may not be necessary here}
\note{Should refer to previous works more here}
Here, we refer to conventional terms used to describe agent environments, described in \cite[p.~41]{AIAMA}. The agent's environment is \textit{partially observable}, since it is assumed that it cannot directly observe the location of the target, but must instead use partial information related to the location of the target from noisy sensors. The outcomes of the agent's actions are assumed to be \textit{deterministic}, meaning that if an agent chooses to move to a location, it is assumed to do so without any chance of it accidentally moving to an alternative location. The environment is \textit{sequential}, arising from the fact that future decisions on where the agent should take a sensor reading are influenced by previous locations at which a sensor reading has been taken. The agent is assumed to operate in a 2-dimensional environment, consisting of discrete uniformly spaced grid cells overlaid onto a physical region of space.
%The environment state can then defined by the tuples of the unknown location of the target with the search status.
The unknown location of the target can be described by the set
\[\{x_1, x_2, ..., x_n, x_{n+1}\}\]
where $x_i$ represents the target location being at grid cell $i$ for $i \in \{1, 2, .., n\}$, and $x_{n+1}$ represents that the target is not present. The search status can be described by 
\[ \{ongoing, terminated\_x_1, terminated\_x_2, ..., terminated\_x_n, terminated\_x_{n+1}\} \]
where ongoing represents that the search is continuing and $terminated\_x_i$ is an absorbing terminal state that arises from the agent taking a terminal action indicating the target location, explained further in the subsequent paragraph. It is necessary to include the terminal states in the environment representation in order to specify a \textit{performance measure} for the agent. 
%For technical reasons, the agent's location was also included in the environment state.
The Cartesian product of these sets defines the environment state. The graphical model shown in Figure \ref{fig:FirstDBNUsed} depicts the conditional independence assumptions made between the hidden state variables.
\note{Goal state not explicitly mentioned - might be worth explicitly stating.}