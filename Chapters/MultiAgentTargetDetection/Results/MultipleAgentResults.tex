\subsection{Varying number of search agents}

This experiment explored how varying the number of agents involved in the search impacted on the effectiveness of the search. We assumed that agents communicated their most recent observation to all other agent at each time step, which the other agents used to update their local belief that the target it present in the region.


\begin{table}[H]
    \centering
    \begin{tabular}{| >{\centering} m{18mm} | >{\centering}m{20mm} | >{\centering}m{18mm} | >{\centering}m{20mm} | >{\centering}m{20mm} | m{20mm} <{\centering}|}
    \hline
       Strategy & \# Agents & Mean TTD & Sample SD[TTD] & False Negative Rate & Proportion Incorrectly Localised \\
        \hline
        $\epsilon$-Greedy & 1 & 112.9258 & 62.3798 & 0.1516 & 0.0398 \\
        $\epsilon$-Greedy & 2 & 65.5912 & 34.3248 & 0.1568 & 0.0314 \\
        $\epsilon$-Greedy & 3 & 47.5176 & 24.9861 & 0.1428 & 0.0262 \\
        \hline
        Sweep & 1 & 601.5697 & 183.4529 & 0.1254 & 0.0454 \\
        Sweep & 2  & 303.7328 & 94.2748 & 0.1232 & 0.0466 \\
        Sweep & 3 & 204.8172 & 65.1273 & 0.1252 & 0.0408 \\
        \hline
        Saccadic & 1 & 98.8274 & 56.1298 & 0.1588 & 0.0370 \\
        Saccadic & 2 & 75.3466 & 39.9718 & 0.1520 & 0.0132 \\
        Saccadic & 3 & 65.0774 & 33.9798 & 0.1598 & 0.0090 \\
        \hline
        Random & 1 & 629.5462 & 282.9514 & 0.1368 & 0.0366 \\
        Random & 2 & 315.0082 & 140.3954 & 0.1254 & 0.0366  \\
        Random & 3 & 211.4242 & 94.7801 & 0.1222 & 0.0448\\
        \hline
    \end{tabular}
   \caption{Results of running the target localisation simulation with a varying number of homogeneous agents for each implemented search strategy.}
    \label{table:VaryingNumberOfAgents}
\end{table}

Table \ref{table:VaryingNumberOfAgents} shows the results of running the search with 1, 2 and 3 homogeneous agents. Each agent is initialised with the same parameters, but start at independently chosen random locations. They receive new observations from other agents at each time-step and then record their own observation. We did not assume that there could be corrupted/interrupted transmissions, although this could be investigated in future work.
%The results agree closely with \cite{Chung2008Multi-agentFramework}. 
Figure 2 and Figure 3 of \cite{Chung2008Multi-agentFramework} show sample runs of adaptive and non-adaptive search strategies for the same simulation configuration. The observed results matched our expectations based on \cite{Chung2008Multi-agentFramework}, where the largest relative reduction in the mean TTD is found in the non-adaptive cases. This is because the non-adaptive cases use unbiased sampling methods. The sweep search method samples all grid cells an equal number of times, so running multiple agents is equivalent to running a single agent with the non-adaptive strategy that can sample multiple times on each time-step rather than just once. This implies the search time is inversely proportional to the number of agents used in the non-adaptive cases, supported by Table \ref{table:VaryingPriorDistribution}. Since the agents first record a sensor observation and update their belief before updating their belief based on other agent observations made at the current time-step, sometimes an agent will end its search before the other agents, which means that the effect of having multiple agents is lessened. This is more prevalent when using the adaptive strategies since when the agents locate the source correctly, there is a chance that one could receive a false negative, while the others terminate with true positive observations. This leads the last agent to make extra observations, inflating the mean TTD. The unbiased methods do not exhibit this behaviour as much and usually terminate when one agent at the target location makes a positive observation which is shared among other agents who are not at the target location and who then also subsequently terminate.

