

\workinprogress

At each discrete timestep, the agent can either choose to move to a new grid location to record a sensor measurement or it can decide to terminate the search based on its estimated state of the environment. There is a trade-off in terminating the search early, which means that less time and resources are spent on continuing the search, versus the possibility of drawing misinformed conclusions from the search due to a lack of information. For example, if the agent receives a series of false positive readings at a given location, it could mistakenly choose to conclude that the target is present at a given location rather than sample further to gain confidence that it has correctly found the location of the target. Following this line of thinking, it is clear that a strategy needs to be devised to minimize the probability of drawing false conclusions, which is in line with the performance measure set out in \ref{sssection:PerfMeas}.\par
Previous related work, \cite{Chung2007ASearchb} has addressed this problem using methods that use heuristics as well as a more formal asymptotic theory-based approach. We ultimately choose to implement the Sequential Probability Ratio Test (SPRT), which is a hypothesis-testing framework developed by \citeauthor{Wald1950BayesProblems} to optimally deal with sequential decision problems, as opposed to traditional frameworks which assume that all the necessary data has been gathered prior to analysis \cite{Wald1950BayesProblems}. The details of the proof of optimality of the SPRT is given in \cite{Wald1950BayesProblems} and we have outlined the details of how to perform hypothesis-testing using this framework in section <refer to the section>, along with the practical advantages and drawbacks of using it. In order to allow the agent to make a decision on whether to terminate the search or not, the following procedure was used: \note{Might be worthwhile simply outlining the algorithm}
\begin{gather}\label{eqn:SearchStatus}
H_0 : \text{The target is not present in the search region}
\\
H_1 : \text{The target is present in the search region}
\end{gather}