In order to add high-level autonomous aerial vehicle support to the simulation environment, we used the AirSim \cite{ShahAirSim:Vehicles} plugin for UE4. We chose to use AirSim in the simulation environment for the following reasons: 
\begin{itemize}
\item AirSim has high-level navigation and sensor APIs for C++, Python and C\#.
\item AirSim provides the ability to run multiple RAVs in the same environment.
\item AirSim has been open-sourced with an MIT licence and was designed to be easily extended.
\item AirSim  provides an API to render unmodified images, depth maps and segmented images.
\item AirSim is highly modular in its approach to modelling sensors and actuators, which means that it can easily be extended with domain-specific functionality.
\end{itemize}
AirSim internally models RAVs using different low-level modules outlined in \cite{ShahAirSim:Vehicles}. Details of how to integrate AirSim into a custom UE4 project can be found here.  \note{It might be worth discussing how airsim actually models the state of the drone and its ability to integrate with PX4 over mavlink}. Once the plugin had been successfully integrated into the project, some additional functionality was added to meet some of the design objectives, described in the subsequent section.

\subsubsection{Modifications and Extensions}
